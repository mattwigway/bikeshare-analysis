\documentclass[11pt]{article}
\title{Predicting the Popularity of Bicycle-Sharing Systems: An Accessibility-Based Approach}
\author{Matthew Wigginton Conway}
\date{\today}
\usepackage[utf8]{inputenc}

\usepackage[authordate,strict,sorting=nyt,babel=other,cmsdate=both]{biblatex-chicago}
\bibliography{BikeShare}

\begin{document}

Bikesharing systems have become popular of late, being installed in
many cities. They consist of electronic stations distributed
at regular intervals throughout a city. Members of the system can
check out bikes at any station and return them to any other
station. Some stations are, of course, more popular than others. It
would seem logical to suspect that stations near jobs or housing are
more likely to be popular. That is, one would expect to find that
stations with the highest accessibility to jobs and housing are the
most popular. Station popularities are spatially autocorrelated
\autocite[5--8]{Conway2013}. This article explores whether
measures of accessibility can be used to explain and predict
bikeshare popularity and explain the autocorrelation seen in
popularities. A model of bikeshare popularity is first developed in
Washington, DC (Capital Bikeshare), and then transferred to San
Francisco (Bay Area Bikeshare) and Minneapolis (Nice Ride Minnesota).

\section{Data Acquisition}

Five variables were calculated to use as input to the model: the
number of jobs within 60 minutes by transit of each station, the
population within 60 minutes by transit, the population and number of
jobs within 10 minutes by walking, and the number of bikeshare
stations within 30 minutes by cycling. Bikeshare is often combined
with transit in multimodal trips \autocite[29]{CABI2013}.
